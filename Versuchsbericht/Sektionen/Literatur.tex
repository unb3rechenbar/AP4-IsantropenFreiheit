\documentclass[../main.tex]{subfiles}
\begin{document}
	\begin{thebibliography}{9}
		\bibitem{skript} Bernd-Uwe Runge (2020) \emph{Physikalisches Anfängerpraktikum der Universität Konstanz}, Universität Konstanz. last compiled on 13. Mai 2022 at 7:11. 

		\bibitem{nolting42} Nolting, W. (2016). \emph{Grundkurs Theoretische Physik 4/2: Thermodynamik}. 9. Auflage. Springer Spektrum. ISBN 978-3-662-49033-4.
		
		\bibitem{heintze} Joachim Heintze. (2016). \emph{Lehrbuch zur Experimentalphysik 2: Kontinuumsmechanik und Thermodynamik}. Springer Spektrum. ISBN 978-3-662-45767-2.

		\bibitem{nist:gasconstant} National Institute of Standards and Technology (2022). \emph{Molar gas constant}. URL \url{https://physics.nist.gov/cgi-bin/cuu/Value?r|search_for=gas+constant}, last visited on 2022-05-13.

		%\bibitem{Unsicherheiten} Philipp Möhrke, Bernd-Uwe Runge  (2020) \emph{Arbeiten mit Messdaten Eine praktische Kurzeinführung nach GUM}, Springer Spektrum  Berlin. ISBN 978-3-662-60659-9.
		
        \bibitem{demtroeder1.9} Wolfgang Demtröder (2021) \emph{Experimentalphysik - Mechanik und Wärme}, 9. Auflage, Springer-Verlag. ISBN 978-3-662-62727-3.
  
        %\bibitem{demtroeder2.7} Wolfgang Demtröder (2017) \emph{Experimentalphysik 2 - Elektrizität und Optik}, 7. Auflage, Springer-Verlag. ISBN 978-3-540-79294-9.
		
		%\bibitem{Metzler2} Grehn, Joachim und von Hessberg, Albrecht und Holz, Hans-Gerd und Krause, Joachim und Krüger, Herwig und Schmidt, Hans Kurt (1979). \emph{Metzler Physik}. 2. Auflage, Druck von 1990. J.B.Metzlersche Verlagsbuchhandlung und Carl Ernst Poeschel Verlag GmbH, Stuttgart. ISBN 3-476-50209-0.
		
		%\bibitem{Metzler3} Bolz, Dr. Joachim und Grehn, Joachim und Krause, Joachim und Krüger, Herwig und Kurt Schmidt, Dr. Herbert und Schwarze, Dr. Heine (1998). \emph{Metzler Physik}. 3. Auflage. Schroedel Verlag GmbH, Hannover. ISBN 3-507-10700-7. 
		
		%\bibitem{BiotSarvart} Wikipedia (2022). \emph{Biot-Savart-Gesetz --- Wikipedia{,} die freie Enzyklopädie}. URL \url{https://de.wikipedia.org/w/index.php?title=Biot-Savart-Gesetz&oldid=220488347}. Abgerufen am 28.05.2022. 
		
		%\bibitem{Helmholtz} Wikipedia (2022). \emph{Magnetische Feldstärke --- Wikipedia{,} die freie Enzyklopädie}. URL \url{https://de.wikipedia.org/w/index.php?title=Magnetische_Feldstärke&oldid=222529257}. Abgerufen am 28.05.2022. 
		
		%\bibitem{MagnetfeldHelmholtz} Universaldenker (2022). \emph{Herleitung Magnetfeld einer Helmholtz-Spule.} URL: \url{https://de.universaldenker.org/argumentationen/326}. Abgerufen am 29.05.2022. 
		
		%\bibitem{HelmholtzDatenSite} Yumpu - Gebrauchsanweisung 555 57/58/59 \emph{Daten der verwendenten Helmoltzspulen.} URL: \url{https://www.yumpu.com/de/document/read/40278843/gebrauchsanweisung-555-57-58-59-instruction-sheet-}. Abgerufen am 30.05.2022. 
		
		%\bibitem{Pischel14} Sören Pischel (2014), Bestimmung der Spezifischen Elementarladung mit Radioröhren, \emph{Bachelorarbeit zur Erlangung des akademischen Grads, Universität Bielefeld}.
		
		%\bibitem{NatJournal} Nature Journal (2014). \emph{High-precision measurement of the atomic mass of the electtron}. URL \url{https://www.nature.com/articles/nature13026}. Abgerufen am 31.05.2022. 
		
		%\bibitem{WorldDoc} World Documents (201). \emph{Das Beta-Spektrometer}. URL \url{https://fdocuments.net/document/das-beta-spektrometer.html}. Abgerufen am 01.06.2022. 
		
		%\bibitem{NIST} The NIST Reference on Constants, Units, and Uncertainty. \emph{elementary charge}. URL \url{https://physics.nist.gov/cgi-bin/cuu/Value?e}. Abgerufen am 04.06.2022.
		
		%\bibitem{Studyflix} Membranpotential - einfach erklärt. \emph{Studyflix}. URL \url{https://studyflix.de/biologie/membranpotential-2808}. Abgerufen am 04.06.2022.
	\end{thebibliography}
\end{document}