\documentclass{subfiles}

\begin{document}
    \subsection{States of matter}
        % Sis is se wörst aidea eva

        Matter in general is found in three states, named \emph{solid}, \emph{liquid} and \emph{gaseous}. We will focus on the latter two, as they are the most relevant for our experiment. First we discuss the so called \emph{ideal Gas} in the following section. 

        \subsubsection*{Ideal Gas}
            An ideal Gas is characterized by the properties of molecules having \emph{no volume} and \emph{no interaction} with each other. This means that the molecules are considered to be point-like. This assumption leads to the most fundamental equation in this experiment, the \emph{ideal gas law}:
            \[
                pV = nRT,
            \]
            where $p$ is the pressure, $V$ the volume, $n$ the number of moles, $R$ the universal gas constant and $T$ the temperature ´\footnote{The universal gas constant is given by $R = \SI{8.314}{\joule\per\mole\per\kelvin}$, found in \cite{nist:gasconstant}.} \cite[p.11]{nolting42}. 

        \subsubsection*{Real Gas}
            The introduced assumptions in ideal gas theory are in practice only met when infinitely decreasing density. \emph{Infinitely} in reality is not achievable, so the assumptions need a refresh. We assign every gasmolecule a volume $V_m$ and a force $F_m$ resulting interaction with close molecules. Doing this we enter the realm of \emph{real gas theory}, in which multiple models exist. One of them is the \emph{van der Waals} model, in which coefficients $a$ and $b$ are introduced to account for the volume and interaction of the molecules. The van der Waals equation is then given by
            \[\nbra{p + \frac{a\cdot n^2}{V^2}} \cdot \nbra{V - n\cdot b} = nRT.\]
            The coefficients $a$ and $b$ are specific to each type of gas. \cite[p.13]{nolting42}
    
    \subsection{Thermodynamic aspects}
        Some of the typical thermodynamic terms were already used above. We now want to introduce them in a more formal way.
        \subsubsection*{State variables}
            A so called \emph{state variable} in general is a variable that \emph{describes} the state of a system. In thermodynamics of gases, the most common state variables are \emph{pressure} $p$, \emph{volume} $V$, \emph{temperature} $T$ and \emph{number of moles} $n$ or \emph{total number of molecules} $N$ \cite[p.5]{nolting42}. While those are mostly self explanatory, the next state variables called \emph{inner energy} $U$ and \emph{entropy} $S$ needs some understanding. To do so, we need to introduce the concept of \emph{thermodynamic processes}.
        \subsubsection*{Thermodynamic processes}
            A thermodynamic process is a process in which a system changes its state variables. The first law of thermodynamics is an example of such. It states that the change of inner energy $dU$ from a state $x\in\Def{U}$ to a new state $x + h$ is equal to the \emph{heat} $dQ$ added to the system minus the \emph{work} $dW$ done by the system:
            \[dU(x)(h) = dQ(x)(h) + dW(x)(h),\quad h\in\Def{U}.\]
            We now found a description of $U$ using its derivative \cite[p.36]{nolting42}. For the entropy $S$ we use the introduced term \emph{heat} and its derivative: $dS(x)(h) = dQ(x)(h) / dT(x)(h)$ \cite[p.56]{nolting42}. 

        \subsubsection*{Isentropic exponent}
            In an \emph{adiabatic process} no heat is added to the system. This means that $dQ(x)(h) = 0$ for every state $x$ and change $h$. Using the first law of thermodynamics and assuming ideal gas we can write
            \[dT(x)(V) = -\frac{p}{C_V} = \frac{n\cdot R}{C_V}\cdot\frac{T}{V},\]
            where we first used $dU(x)(V) = 0$ and $p = nRT/V$ and introduced $C_V$ as the \emph{thermal capacity} at constant volume. Using $nR = C_p-C_V$ and introducing $C_p$ as the \emph{thermal capacity} at constant pressure we find
            \[dT(x)(V) = \frac{C_p-C_V}{C_V} \cdot\frac{T}{V}.\]
            Now we can define the \emph{isentropic exponent} $\kappa$ as $\kappa:=C_P/C_V$ and find the \emph{adiabatic equation}
            \[T\cdot V^{1-\kappa} = \textit{const}. \Longleftrightarrow p\cdot V^\kappa = \textit{const}.\]

        \subsubsection*{Poisson equations}
            The \emph{poisson equation} in general is a partial differential equation of the form
            \[-\lap u(t) = F(t,u(t))\]
            with some right side $F\in C^0(\R\times\R^d)$ and the sought function $u\in C^2(\R^d)$. 

        \subsubsection*{Thermal capacity}


% ideal gas
% state variables + thermodynamic processes
% 1. law of thermodynamics
% isentropic exponent + Poisson équations: show connections to degrees of (but no derivation of Poission equations required, good for us)
% thermal capacity
\end{document}