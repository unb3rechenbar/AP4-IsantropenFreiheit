\documentclass{subfiles}

\begin{document}
    \subsection{States of matter}
        % Sis is se wörst aidea eva

        Matter in general is found in three states, named \emph{solid}, \emph{liquid} and \emph{gaseous}. We will focus on the latter two, as they are the most relevant for our experiment. First we discuss the so called \emph{ideal Gas} in the following section. 

        \subsubsection*{Ideal Gas}
            An ideal Gas is characterized by the properties of molecules having \emph{no volume} and \emph{no interaction} with each other. This means that the molecules are considered to be point-like. This assumption leads to the most fundamental equation in this experiment, the \emph{ideal gas law}:
            \[
                pV = nRT,
            \]
            where $p$ is the pressure, $V$ the volume, $n$ the number of moles, $R$ the universal gas constant and $T$ the temperature\footnote{The universal gas constant is given by $R = \SI{8.314}{\joule\per\mole\per\kelvin}$. }
        \subsubsection*{Real Gas}
    
    \subsection{Thermodynamic aspects}
    
        \subsubsection*{State variables}

        \subsubsection*{Thermodynamic processes}

        \subsubsection*{First law of thermodynamics}

        \subsubsection*{Isentropic exponent}

        \subsubsection*{Poisson equations}

        \subsubsection*{Thermal capacity}


% ideal gas
% state variables + thermodynamic processes
% 1. law of thermodynamics
% isentropic exponent + Poisson équations: show connections to degrees of (but no derivation of Poission equations required, good for us)
% thermal capacity
\end{document}