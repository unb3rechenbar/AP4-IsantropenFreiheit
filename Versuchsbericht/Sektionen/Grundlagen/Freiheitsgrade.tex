\documentclass{subfiles}

\begin{document}

    \subsection{Degrees of freedom}
        The degrees of freedom of a system are the number of independent variables needed to describe the system fully.

        \subsubsection*{Translation, rotation, vibration}
        Single atoms only have three \textit{translational} degrees of freedom, which are their x-,y-, and z-coordinates. More complex molecules may have different states depending on their rotation, making for up to three additional \textit{rotational} degrees of freedom. For example, $O_2$ has two rotational degrees of freedom, since its symmetrical along the axis connecting the atoms.Lastly, a molecule can also vibrate. Here, the \textit{vibrational} degrees of freedom are double the number of eigen-vibrationmodes. The fact that each mode is counted twice makes for a more convenient mathematical treatment of thermodynamic equations.\\

        \noindent Due to quantummechanical effects, not every degree of freedom may be available to a system with a given energy. Hence the number of \textit{effective} degrees of freedom $f_{eff}$ is smaller than the actual number $f$ \cite[p.75-76]{heintze}.\\

        \noindent In statistical mechanics it is shown, that the inner energy $U$ of one mole is connected to the degrees of freedom as follows:
        \begin{align*}
            U=\frac{1}{2}\cdot f\cdot N_A\cdot kT=\frac{1}{2}\cdot(f_{trans}+f_{rot}+f_{vibr})\cdot N_A\cdot kT.
        \end{align*}
        At constant volume the change heat $\Delta Q$ is equal to $\Delta U$ and thus the above equation gives
        \begin{align*}
            \Delta Q=\frac{1}{2}\cdot f\cdot N_Ak\cdot\Delta T=\frac{f\cdot R}{2}\cdot\Delta T\implies C_V=\frac{1}{2}\cdot f\cdot R
        \end{align*}
        for the thermal capacity $C_V$. At constant pressure, the ideal gas law gives $\Delta Q=\Delta U+p\cdot\Delta V$ and furthermore
        \begin{align*}
            \Delta Q=(C_V+R)\cdot\Delta T\implies C_p=C_V+R.
        \end{align*}
        Using the relation $\kappa=C_p/C_V$ to the isentropic exponent it follows
        \begin{align*}
            \kappa=\frac{f+2}{f},
        \end{align*}
        which can be used to determine the effective degrees of freedom of a given substance \cite[p.296-297]{demtroeder1.9}.

        \subsubsection*{Standing waves and resonance}

        \subsubsection*{Forced oscillations}
c
    \subsection{Sound}

        \subsubsection*{Velocity of sound}

        \subsubsection*{Kirchhoff correction}

        \subsection{Experimental principle}

        \subsubsection*{Rüchhardt and Flammersfeld}

        \subsubsection*{Clément and Desormes}


% degrees of freedom: translation, rotation vibration
% standing waves + resonance condition in resonance tube
% forced oscillations
% velocity of sound in matter, including Kirchhoff correction
% experimental principle: determination of isentropic exponent: measurement according to Rüchhardt and Flammersfeld, measurement according to Clément and Desormes

\end{document}